
\newcommand{\mypage}[1]{
    \begin{minipage}[t]{0.4\textwidth}
        {#1}
    \end{minipage}
}

%	\setlength{\arrayrulewidth}{0.5mm}
%	\setlength{\tabcolsep}{18pt}
\newcommand{\programme}{
    \renewcommand{\arraystretch}{2}
    \begin{center}
        \begin{tabular}{@{}|l|l|@{}}
            \multicolumn{2}{c}{
            Programme : Un niveau d’organisation, les éléments chimiques} \\ \midrule

            \mypage{Savoirs} & \mypage{Savoirs-faire}                     \\\midrule

            \mypage{
                Les noyaux des atomes de la centaine
                d’éléments chimiques stables résultent de
                réactions nucléaires qui se produisent au sein des étoiles à partir de l’hydrogène initial.

                La matière connue de l’Univers est formée principalement
                d’hydrogène et d’hélium alors que la Terre est surtout constituée
                d’oxygène, d’hydrogène, de fer, de silicium, de magnésium
                et les êtres vivants de carbone, hydrogène, oxygène et
                azote.}
                             &

            \mypage{
                Produire et analyser différentes
                représentations graphiques de l’abondance des éléments chimiques (proportions)
                dans l’Univers, la Terre, les êtres vivants.
                L’équation d’une réaction nucléaire stellaire étant
                fournie, reconnaître si celle-ci relève d’une fusion ou
                d’une fission.
            }                                                             \\

            \mypage{Certains noyaux sont instables et se
                désintègrent (radioactivité).
                L’instant de désintégration d’un noyau
                radioactif individuel est aléatoire.
                La demi-vie d’un noyau radioactif est la durée nécessaire pour que la moitié des noyaux initialement présents dans un échantillon macroscopique se soit désintégrée.
                Cette demi-vie est caractéristique du noyau radioactif.
            }
                             &
            \mypage{Calculer le nombre de noyaux restants au
                bout de n demi-vies
                Estimer la durée nécessaire pour obtenir une certaine proportion de noyaux restants.
                Utiliser une représentation graphique pour déterminer une demi-vie.
                Utiliser une décroissance radioactive pour
                une datation (exemple du carbone 14).
            }                                                             \\
            \bottomrule\end{tabular}
    \end{center}

}

