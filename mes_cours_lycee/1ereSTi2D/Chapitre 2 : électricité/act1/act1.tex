%%%%%%%%%%%%%%%%%%%%%%%%%%%%%%%%%%%%%%%%%
% Cleese Assignment (For Students)
% LaTeX Template
% Version 2.0 (27/5/2018)
%
% This template originates from:
% http://www.LaTeXTemplates.com
%
% Author:
% Vel (vel@LaTeXTemplates.com)
%
% License:
% CC BY-NC-SA 3.0 (http://creativecommons.org/licenses/by-nc-sa/3.0/)
% 
%%%%%%%%%%%%%%%%%%%%%%%%%%%%%%%%%%%%%%%%%

%----------------------------------------------------------------------------------------
%	PACKAGES AND OTHER DOCUMENT CONFIGURATIONS
%----------------------------------------------------------------------------------------

\documentclass[10pt]{article}
%%%%%%%%%%%%%%%%%%%%%%%%%%%%%%%%%%%%%%%%%
% Cleese Assignment
% Structure Specification File
% Version 1.0 (27/5/2018)
%
% This template originates from:
% http://www.LaTeXTemplates.com
%
% Author:
% Vel (vel@LaTeXTemplates.com)
%
% License:
% CC BY-NC-SA 3.0 (http://creativecommons.org/licenses/by-nc-sa/3.0/)
% 
%%%%%%%%%%%%%%%%%%%%%%%%%%%%%%%%%%%%%%%%%

%----------------------------------------------------------------------------------------
%	PACKAGES AND OTHER DOCUMENT CONFIGURATIONS
%----------------------------------------------------------------------------------------

\usepackage{lastpage} % Required to determine the last page number for the footer


\usepackage{graphicx} % Required to insert images

\setlength\parindent{0pt} % Removes all indentation from paragraphs

\usepackage[svgnames]{xcolor} % Enabling colors by their 'svgnames'
\usepackage[most]{tcolorbox} % Required for boxes that split across pages

\usepackage{booktabs} % Required for better horizontal rules in tables

\usepackage{listings} % Required for insertion of code

\usepackage{etoolbox} % Required for if statements

\usepackage{multido} % required for dotted lines

% \usepackage{arydshln} % Required for dotted lines
% \setlength{\dashlinedash}{0.2pt}
% \setlength{\dashlinegap}{4.5pt}
% \setlength{\arrayrulewidth}{0.2pt}


\usepackage[french]{babel} % English language hyphenation

%----------------------------------------------------------------------------------------
%	MARGINS
%----------------------------------------------------------------------------------------

\usepackage{geometry} % Required for adjusting page dimensions and margins

\geometry{
	paper=a4paper, % Change to letterpaper for US letter
	top=    2.5cm, % Top margin
	bottom= 2.5cm, % Bottom margin
	left= 1.cm, % Left margin
	right=1.cm, % Right margin
	headheight=14pt, % Header height
	footskip=1.4cm, % Space from the bottom margin to the baseline of the footer
	headsep=1.2cm, % Space from the top margin to the baseline of the header
	%showframe, % Uncomment to show how the type block is set on the page
}


% \usepackage{hyperref}
% \hypersetup{%
%   colorlinks=false,% hyperlinks will be black
%   linkbordercolor=blue,% hyperlink borders will be red
%   pdfborderstyle={/S/U/W 1}% border style will be underline of width 1pt
% }

%----------------------------------------------------------------------------------------
%	FONT
%----------------------------------------------------------------------------------------
\usepackage{fontspec}
\usepackage{unicode-math}
\usepackage[utf8]{inputenc} % Required for inputting international characters
\usepackage[T1]{fontenc} % Output font encoding for international characters

% \usepackage[sfdefault,light]{roboto} % Use the Roboto font
\setmainfont[Ligatures=TeX]{Caladea}
%----------------------------------------------------------------------------------------
%	HEADERS AND FOOTERS
%----------------------------------------------------------------------------------------

\usepackage{fancyhdr} % Required for customising headers and footers

\pagestyle{fancy} % Enable custom headers and footers

\lhead{\small\assignmentClass\ifdef{\assignmentClassInstructor}{\ (\assignmentClassInstructor):}{}\ \assignmentTitle} % Left header; output the instructor in brackets if one was set
\chead{} % Centre header
\rhead{\small\ifdef{\assignmentAuthorName}{\assignmentAuthorName}{\ifdef{\assignmentDueDate}{Due\ \assignmentDueDate}{}}} % Right header; output the author name if one was set, otherwise the due date if that was set

\lfoot{} % Left footer
\cfoot{\small Page\ \thepage\ sur\ \pageref{LastPage}} % Centre footer
\rfoot{} % Right footer

\renewcommand\headrulewidth{0.5pt} % Thickness of the header rule


\fancypagestyle{plain}{
	\lhead{} % Left header; output the instructor in brackets if one was set
	\chead{} % Centre header
	\rhead{}

  \renewcommand{\headrulewidth}{0pt}%
  \fancyfoot[C]{\small Page\ \thepage\ sur\ \pageref{LastPage}}%
}

%----------------------------------------------------------------------------------------
%	MODIFY SECTION STYLES
%----------------------------------------------------------------------------------------

\usepackage{titlesec} % Required for modifying sections

%------------------------------------------------
% Section

\titleformat
{\section} % Section type being modified
[runin] % Shape type, can be: hang, block, display, runin, leftmargin, rightmargin, drop, wrap, frame
{\large} % Format of the whole section
{
\textbf{\assignmentQuestionName~\thesection :}	
} % Format of the section label
{6pt} % Space between the title and label
{} % Code before the label

\titlespacing{\section}
{-10pt}
{0.3\baselineskip}
{0.3\baselineskip} % Spacing around section titles, the order is: left, before and after

%------------------------------------------------
% Subsection

\titleformat
{\subsection} % Section type being modified
[block] % Shape type, can be: hang, block, display, runin, leftmargin, rightmargin, drop, wrap, frame
{\itshape \large} % Format of the whole section
{(\alph{subsection})} % Format of the section label
{4pt} % Space between the title and label
{} % Code before the label

\titlespacing{\subsection}
{0pt}
{0.3\baselineskip}
{0.3\baselineskip} % Spacing around section titles, the order is: left, before and after

\renewcommand\thesubsection{(\alph{subsection})}

%----------------------------------------------------------------------------------------
%	CUSTOM QUESTION COMMANDS/ENVIRONMENTS
%----------------------------------------------------------------------------------------

% Environment to be used for each question in the assignment
\newenvironment{question}{
	\vspace{0.2\baselineskip} % Whitespace before the question
	\section{} % Blank section title (e.g. just Question 2)
	\lfoot{La \small\itshape\assignmentQuestionName~\thesection~continue sur la page d'après\ldots} % Set the left footer to state the question continues on the next page, this is reset to nothing if it doesn't (below)
}
{
	\lfoot{} % Reset the left footer to nothing if the current question does not continue on the next page
}

%------------------------------------------------

% Environment for subquestions, takes 1 argument - the name of the section
\newenvironment{subquestion}[1]{
	\subsection{#1}
}{
}

%------------------------------------------------

% Command to print a question sentence
\newcommand{\questiontext}[1]{
	\large #1
	\vspace{0.2\baselineskip} % Whitespace afterwards
}

%------------------------------------------------

% Command to print a box that breaks across pages with the question answer
\newcommand{\answer}[1]{
	\begin{tcolorbox}[breakable, enhanced]
		#1
	\end{tcolorbox}
}

%------------------------------------------------

% Command to print a box that breaks across pages with the space for a student to answer
\newcommand{\answerbox}[1]{
	\begin{tcolorbox}[breakable, enhanced]
		% \vphantom{L}
		% \vspace{\numexpr #1-1\relax\baselineskip} % \vphantom{L} to provide a typesetting strut with a height for the line, \numexpr to subtract user input by 1 to make it 0-based as this command is
		\vspace{15pt}
		\multido{}{#1}{\noindent\makebox[\linewidth]{\dotfill}
		\endgraf\vspace{12pt}}% ... dotted lines ...
		\vspace{-13pt}
	\end{tcolorbox}
}


\newtcolorbox{mybox}[1][]{
    colbacktitle=blue!10!white,
    colback=red!10!white,
    coltitle=blue!70!black,
    adjusted title={#1},
    fonttitle=\bfseries,
    % attach title to upper={\ --- \ }
    }

%------------------------------------------------

% Command to print an assignment section title to split an assignment into major parts
\newcommand{\assignmentSection}[1]{
	{
		\centering % Centre the section title
		\vspace{1\baselineskip} % Whitespace before the entire section title
		
		\rule{0.8\textwidth}{0.5pt} % Horizontal rule
		
		\vspace{0.75\baselineskip} % Whitespace before the section title
		{\LARGE \MakeUppercase{#1}} % Section title, forced to be uppercase
		
		\rule{0.8\textwidth}{0.5pt} % Horizontal rule
		
		\vspace{\baselineskip} % Whitespace after the entire section title
	}
}


%	TITLE SECTION
%----------------------------------------------------------------------------------------

% \newcommand{\authorstyle}[1]{{\large\usefont{OT1}{phv}{b}{n}#1}} % Authors style (Helvetica)

% \newcommand{\institution}[1]{{\footnotesize\usefont{OT1}{phv}{m}{sl}#1}} % Institutions style (Helvetica)

\usepackage{titling} % Allows custom title configuration

\newcommand{\HorRule}{\rule{\linewidth}{1pt}} % Defines the gold horizontal rule around the title

\pretitle{
	\centering
	\vspace{-120pt} % Move the entire title section up
	\HorRule\vspace{10pt} % Horizontal rule before the title
	% \textbf
	\bfseries
	\fontsize{32}{36}
	\usefont{T1}{phv}{b}{n}
	\selectfont % Helvetica
	\color{DarkRed} % Text colour for the title and author(s)
}

\posttitle{\par\vskip 15pt} % Whitespace under the title

\preauthor{} % Anything that will appear before \author is printed

\postauthor{ % Anything that will appear after \author is printed
	% \vspace{10pt} % Space before the rule
	\par\HorRule % Horizontal rule after the title
	\vspace{-40pt} % Space after the title section
}

\usepackage{blindtext, subfig}
\usepackage{dblfloatfix} % fix for bottom-placement of figure
\usepackage{adjustbox}





\usepackage{tikz}
\usepackage{mdframed}

% % 1st arg : nom (cours ou definition)
% % 2eme : this title
% \newcommand{\mymacro}[2]{
% 	\ifstrempty{#2}%
% 	{\mdfsetup{%
% 	frametitle={%
% 		#1 
% 	 }}}
% 	{\mdfsetup{%
% 	frametitle={%
% 		#1 -- #2
% 	 }}}
% }

% \newenvironment{thmarnaud}[1][]{%
   
% 	\mymacro{theoreme}{#1}

% 	\mdfsetup{
% 	innertopmargin=10pt,linecolor=blue!20,%
%       linewidth=2pt,topline=true,
%       frametitleaboveskip=\dimexpr-\ht\strutbox\relax,}
% 	\begin{mdframed}[]\relax%
%       }{\end{mdframed}}



\newcounter{definitioncount}[section]
\newenvironment{definition}[1][]{%
   \stepcounter{definitioncount}%
   \ifstrempty{#1}%
   {\mdfsetup{%
         frametitle={%
               \tikz[baseline=(current bounding box.east),outer sep=0pt]
               \node[anchor=east,rectangle,fill=blue!20]
               {\strut Définition~\thedefinitioncount};}}
   }%
   {\mdfsetup{%
         frametitle={%
               \tikz[baseline=(current bounding box.east),outer sep=0pt]
               \node[anchor=east,rectangle,fill=blue!20]
               {\strut Définition~\thedefinitioncount:~#1};}}%
   }%
   \mdfsetup{
	frametitlealignment=\centering,
	nobreak=true,
	roundcorner=10pt,
	innertopmargin=10pt,
	linecolor=blue!20,
	linewidth=2pt,
	topline=true,
    	frametitleaboveskip=\dimexpr-\ht\strutbox\relax,}
   \begin{mdframed}[]\relax%
      }{\end{mdframed}}


\newenvironment{concept}[1]{
	\mdfsetup{%
	frametitlealignment=\centering,
	nobreak=true,
	roundcorner=10pt,
	frametitle={
		{#1}
		},
		linecolor=red,
		middlelinewidth=2pt,%
		frametitlerule=true,%
		apptotikzsetting={\tikzset{mdfframetitlebackground/.append style={%
		shade,left color=white, right color=blue!20}}},
		frametitlerulecolor=green!60,
		frametitlerulewidth=1pt,
		innertopmargin=\topskip,
	}
	\begin{mdframed}[]\relax%
		}{\end{mdframed}}
  


\newcounter{documentcount}[section]
\newenvironment{documentpeda}[1]{
	\stepcounter{documentcount}%
	\mdfsetup{%
	frametitlealignment=\centering,
	nobreak=true,
	roundcorner=10pt,
	% frametitle={{#1}},
		linecolor=red,
		middlelinewidth=2pt,%
		frametitlerule=true,%
		apptotikzsetting={\tikzset{mdfframetitlebackground/.append style={%
		shade,left color=white, right color=blue!20}}},
		frametitlerulecolor=green!60,
		frametitlerulewidth=1pt,
		innertopmargin=\topskip,
	}
	
	\ifstrempty{#1}%
		{
			\mdfsetup{frametitle={{Document~\thedocumentcount }}}
		}
		{
			\mdfsetup{frametitle={{Document~\thedocumentcount : #1}}}
		}
	\begin{mdframed}[]\relax%
}{\end{mdframed}}
 % Include the file specifying the document structure and custom commands
% \usepackage{activite}
%----------------------------------------------------------------------------------------
%	ASSIGNMENT INFORMATION
%----------------------------------------------------------------------------------------

% Required
\newcommand{\assignmentQuestionName}{Question} % The word to be used as a prefix to question numbers; example alternatives: Problem, Exercise
\newcommand{\assignmentClass}{Physique Chimie} % Course/class
\newcommand{\assignmentTitle}{Activité n°1:} % Assignment title or name
\newcommand{\assignmentAuthorName}{Mme Cercy} %

%----------------------------------------------------------------------------------------
%	VARIABLES
%----------------------------------------------------------------------------------------

\newcommand{\titreActivite}{Activité 1} % titre de l'activité

\begin{document}



La tension aux bornes d’une pile dépend-elle du nombre de dipôles alimentés ?
Pile de 4,5 V ou de 9 V (6LR61) ;
Ampèremètre ;
Voltmètre ;
Rhéostat (ou potentiomètre) de 100 $Ω$ ;
Fils de connexion ;
Interrupteur.

Loi des nœuds, loi des mailles, loi d'Ohm
Générateur de tension : 12 V ;
Plaque de câblage et fils électriques ;
Dipôles ohmiques : R1​ =R3​= 220 Ω et R2​=R4​= 470 $Ω$ ;
2 multimètres.


bilan :

Pile de 4,5 V ou de 9 V (6LR61) ;
2 multimètres.
Rhéostat (ou potentiomètre) de 100 Ω ;
Fils de connexion ;
Interrupteur.
2 × 220 $Ω$
2 × 470 $Ω$
%----------------------------------------------------------------------------------------

%----------------------------------------------------------------------------------------
%	TITLE PAGE
%----------------------------------------------------------------------------------------
\date{}
\title{\titreActivite}
\maketitle % Print the title page


\assignmentSection{Votre mission travail}


\begin{center}
	loi maille nœuds
	Comment expliquer la diminution de luminosité de deux lampes dans un montage en série par rapport à un montage en dérivation ? Cette explication est-elle toujours valable dans le cas d'un générateur limité en intensité ?


	matériel 3 :


	Générateur de tension : 12 V ;
	Plaque de câblage et fils électriques ;
	Dipôles ohmiques : R1​ =R3​= 220 Ω et R2​=R4​= 470 Ω ;
	2 multimètres.

	Reproduire le schéma du montage et flécher les tensions U1​, U2​, U3​ et U4​ avec les bornes des résistances.
	Déterminer le nombre de mailles du circuit, les nommer et donner le nom des noeuds électriques.

	Vérifier la validité de la loi des nœuds à l'aide des intensités mesurées.


	5. Vérifier par le calcul la validité de la loi des mailles.


	6. En utilisant la loi d'Ohm, calculer les valeurs des résistances du circuit et les comparer avec les valeurs marquées sur celles-ci.

\end{center}






\begin{center}

	\begin{itemize}
		\item  Comment expliquer la diminution de luminosité de deux lampes dans un montage en série par rapport à un montage en dérivation ?
		      Cette explication est-elle toujours valable dans le cas d'un générateur limité en intensité ?
	\end{itemize}

	schéma : \includegraphics[width=0.2\columnwidth]{ example-image}


	matériel :
	Générateur de tension : 12 V ;
	Plaque de câblage et fils électriques ;
	Dipôles ohmiques : R1​ =R3​= 220 Ω et R2​=R4​= 470 Ω ;
	2 multimètres.





\end{center}
\begin{center}



	Le bilan de puissance d’une installation électrique permet d’identifier les appareils électriques
	les plus énergivores.

	➜ Comment effectuer un bilan de puissance dans un circuit destiné à alimenter
	les phares à LED (DEL en français) d’une voiture ?

	matériel 2 :
	Générateur de tension continue 5 V ;
	Deux DEL blanches ;
	Deux résistors de résistance 220 Ω ;
	Deux multimètres (voltmètre, ampèremètre, ohmmètre) ;
	Fils de connexion.


	Caractéristiques et protection d’une diode

	Doc. 2
	Schéma du montage à réaliser

	1. Doc. 1 À l’aide de la caractéristique de la diode, indiquer la tension aux bornes de la DEL lorsqu’elle est traversée par un courant maximal de 20 mA.


	2. À l’aide de la loi des mailles et de la loi d’Ohm, calculer la valeur minimale de la résistance de protection à placer en série avec la DEL pour une tension d’alimentation de 5 V.


	3. Câbler le montage avec une seule DEL et sa résistance de protection. Après vérification par le professeur, déterminer la puissance de chaque dipôle dans le circuit.


	4. Trouver une relation entre ces puissances, appelée « bilan de puissance ».

	5. Câbler le montage complet du doc. 2 et refaire les mesures de la question 3.


	6. Trouver une relation mathématique entre ces différentes puissances. Conclure.
\end{center}



\begin{center}

	Émilie utilise une pile de 9 V pour faire briller des lampes associées en dérivation.
	Elle constate que la tension aux bornes de la pile diminue quand le nombre de lampes augmente.

	➜ Qu’est-ce que la caractéristique U=f(I) d’une source réelle de tension ?

	Matériel nécessaire

	Logiciel tableur-grapheur ;
	Pile de 4,5 V ou de 9 V (6LR61) ;
	Ampèremètre ;
	Voltmètre ;
	Rhéostat (ou potentiomètre) de 100 Ω ;
	Fils de connexion ;
	Interrupteur.


	1. Schématiser le montage effectué par Émilie avec une pile et trois lampes.

	2. Doc. 1 On souhaite tracer la caractéristique d’une pile similaire. Câbler le montage présenté en choisissant un calibre adapté pour le voltmètre et l’ampèremètre.

	3. Effectuer les mesures nécessaires de U et de I en faisant varier la valeur de la résistance du rhéostat. Tracer la caractéristique U=f(I) à l’aide d’un logiciel tableur-grapheur.

	1. Schématiser le montage effectué par Émilie avec une pile et trois lampes.
	Dessinez ici


	2. Doc. 1 On souhaite tracer la caractéristique d’une pile similaire. Câbler le montage présenté en choisissant un calibre adapté pour le voltmètre et l’ampèremètre.

	3. Effectuer les mesures nécessaires de U et de I en faisant varier la valeur de la résistance du rhéostat. Tracer la caractéristique U=f(I) à l’aide d’un logiciel tableur-grapheur.
	Lancer le module Geogebra
	Vous devez vous connecter sur GeoGebra afin de sauvegarder votre travail

	4. Modéliser cette caractéristique par une régression linéaire. Relever les valeurs du coefficient directeur de la droite et son ordonnée à l’origine, puis à l’aide du doc. 2 en déduire les valeurs numériques des paramètres E0​ et r de la pile.

	. Quelle est la valeur de l’intensité de court-circuit Icc​ pour le générateur réel (U=0 V dans ce cas) ?




\end{center}


\begin{center}

	Comment évaluer le rendement énergétique d’un dispositif ?

	Matériel
	Générateur de tension continue adapté au monte-charge ;
	Moteur monte-charge avec réducteur (environ 400 tours/minute) et interrupteur marche/arrêt ;
	Potence, ficelle et crochet ;
	Plusieurs masses à suspendre de 100 g ;
	Voltmètre, ampèremètre ;
	Chronomètre ;
	Mètre mesureur.

	1. Exprimer le rendement expérimental du dispositif en fonction de m, U, I, g, z et de la durée Δt de levage.


	2. Proposer un protocole permettant de déterminer la valeur expérimentale du rendement.


	3. Émettre et vérifier des hypothèses sur l’influence de différents facteurs sur le rendement du moteur.

	La charge soulevée a-t-elle une influence sur le rendement énergétique du dispositif ?
\end{center}

%----------------------------------------------------------------------------------------
%	QUESTION 1
%----------------------------------------------------------------------------------------

\begin{question}
	\questiontext{
		Question
	}
\end{question}
\answerbox{2}

\end{document}
